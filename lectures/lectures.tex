\documentclass{scrartcl}
\usepackage[utf8]{inputenc}
\usepackage[T2A]{fontenc}
\usepackage[russian]{babel}
\usepackage{amssymb}
\usepackage{amsmath}
\usepackage{amsthm}
\usepackage{listings}
\newtheorem{theorem}{Теорема}
\newtheorem{definition}{Определение}
\newtheorem{corollary}{Следствие}[theorem]
\newtheorem{lemma}[theorem]{Лемма}
\title{Лекции по алгебре}
\author{Титилин Александр}
\date{}
\begin{document}
    \maketitle
    \section{Линейное пространство, свойства, примеры}
    Есть множество L = $\{a,b,\dots\}$ и есть некоторое поле  $P$
    \begin{definition}
        L называется линейным пространством над полем P, если выполняются следущие условия
        \begin{enumerate}
            \item $\forall a,b \in L ~ \exists ! c : a + b = c$
            \item $\forall a,b,c \in L : (a + b) + c = a + (b + c)$
             \item $\exists  \overline{0} \in L \forall  a \in L a + 0 = a$
             \item $\forall  a\in L \exists  \overline{a} \in L : a +  \overline{a} = \overline{0}$
             \item $a + b = b + a \forall  a,b \in L$
        \end{enumerate}
        И существует операция $\forall  a\in L \forall  \alpha \in P \exists ! d \in L : \alpha a = d$
        \begin{enumerate}
            \item $\alpha , \beta \in  P a \in L = (\alpha + \beta)a = \alpha a + \beta a$
            \item  $\forall  a,b \in L , \forall \alpha \in P \alpha(a + b) = \alpha a + \alpha b $
            \item $\exists  1 \in P 1a = a \forall a \in L$
            \item $\alpha(\beta a) = (\alpha \beta)a \forall a \in L \forall \alpha ,\beta \in P $
        \end{enumerate}
    \end{definition}
    \subsection{Свойства операций}
    \begin{enumerate}
        \item $\exists ! \overline{0}$
            Пусть есть два ноля тогда их сумма может быть и первым и вторым значит они равны.
        \item $\forall  a\in L \exists ! \overline{a}$
             Предполагаем что для какого-то а обратный не единственный, найдется два обратных $\overline{a_1} + a + \overline{a_2}$
             \[
             \overline{a_1} + (a + \overline{a_2}) = a_1 + \overline{0} = a_1
             .\] 
             \[
                 (\overline{a_1} + a) + \overline{a_2} = (a + \overline{a_1}) + \overline{a_{2}} = \overline{0} + \overline{a_2} = \overline{a_2} + \overline{0} = \overline{a_2}
             .\] 
            \item $a , b \in L \exists  c \in L : a  = b + c$ c называется разностью $a,b$
                Докажем что разность существует для любых двух элементов и она единсвенная.
                $(a + \overline{b}) + b = a + (\overline{b} + b) = a + (b + \overline{b}) = a + \overline{0} = a$
            \item $0a = \overline{0} \forall \in L$
                \[
                a + 0a = 1a + 0a = (1 + 0)a = 1a = a
                .\] 
            \item $(-1)a = \overline{a} \forall  a\in L$
                \[
                a + (-1)a = 1a + (-1)a = (1 + (-1))a = 0a = 0
                .\] 
    \end{enumerate}
    \subsection{Примеры}
    \begin{enumerate}
        \item Векторы на плоскости и в пространстве.
        \item Векторы с $n$ координат.
        \item Матрицы из вещ чисел  $n \times m$
        \item  $\{\overline{0}\}$ = L, $P = \mathbb{R}$
        \item $L =R_{+} ~ P = \mathbb{R} ~ a + b = ab ~ \alpha a = a^{\alpha}$
        \item L - множество всех полиномов степени не старше n $(P_{n}(t))$.
        \item L - множество всех функций непрерывных на отрезке от 0 до 1.
    \end{enumerate}
    Мы будем терминологически разделять компклесные и вещественные линейные пространства.
    \begin{definition}
        Есть поле P, повесим над ним два линейных пространства $L, L'$. Эти два линейных пространства
        изоморфны друг другу, если между элементами существует биекция,такая что
         \begin{enumerate}
            \item $(a + b)' = a' + b' \forall  a, b \in L$ Образ суммы равен сумме образов.
            \item $(\alpha a)' = \alpha a', \alpha \in P$
        \end{enumerate}
    \end{definition}
    \section{Линейная зависимость, базис, размерность}
    Рассмотрим $a_1,a_2,\dots,a_{n} \in L$
    \begin{definition}[Линейная комбинация]
         \[
         \alpha_1 a_1 + \alpha_2a_2 + \dots + \alpha_{n}a_{n} = \sum_{i=1}^{n} \alpha_{i} a_{i}
         .\] 
     \end{definition}
     Если линейная комбинация равна нулю, то она называется нулевой или тривиальной.
     \begin{definition}[Линейная независимость]
         Система элементов являются линейно независимыми, если их нулевая линейная комбинация достигается при всех нулевых коэффициентов.
     \end{definition}
     \begin{definition}[Линейная зависимость]
         Если нулевую комбинацию можно получить, имея не нулевые коэффициенты, то она
         линейно зависимая
     \end{definition}
     \subsection{Примеры}
     \begin{enumerate}
         \item Векторы на плоскость $\vec{v_1} = \{0,1\} \vec{v_2} =\{1,0\}$
             \[
             \alpha(1,0) + \beta(0,1) = (\alpha,\beta) = 0 \implies \alpha = 0 \land \beta = 0
             .\] 
     \end{enumerate}
     \subsection{Свойства}
     \begin{enumerate}
         \item Если в системе элементов есть нейтральный элемент, то она линейно зависимая.
        \item Если в системе есть два одинаковых, то она линейно зависимая.
        \item Подномножество совокупность линейно зависимое $\implies$ вся совокупность линейно зависимая.
        \item Все подмножества линейно независимой системы линейно независимые.
     \end{enumerate}
     \begin{theorem}
         Система элементов линейно зависимой $\iff$ хотя бы один из элементов представлял собой линейную комбинацию остальных элементов.
     \end{theorem}
     \begin{proof}
        Необходимость. $\alpha_{i} \neq 0$ $\alpha_{i}a_{i} = -\alpha_1 a_{1} - \dots - \alpha_{n} a_{n}$ 
        Достаточность $a_{i} = \beta_1 a_{1} + \dots + \beta_{n} a_{n}$
     \end{proof}
     \begin{theorem}
         \[
         a_1,\dots,a_{k} \in L, b_1,\dots,b_{m} \in L
         .\] 
         Любой элемент второй системы можно представить как линейную комбинацию первой.
         $m > k \implies $ элементы b линейно зависимы
     \end{theorem}
     \begin{proof}
         \begin{enumerate}
             \item $k = 1$
                  \[
                 b_1 = \alpha_{1} a_{1}
                 .\] 
                 \[
                 b_2 = \alpha_{2}a_1
                 .\] 
                 \[
                 \dots
                 .\] 
                 \[
                 b_{m} = \alpha_{m} a_1
                 .\] 
                 \[
                 -\alpha_2 b_1 + \alpha_1 b_2 + 0b_3 + \dots + 0 b_{m} = 0
                 .\] 
            \item 
                Утверждение теоремы верно для $k - 1$
                 \[
                b_1 = \alpha_{11} a_1 + \alpha_{12} a_{2} + \dots + \alpha_{1k}a_{k}
                .\] 
                \[
                b_{i} = a_{i1}a_1 + \dots a_{ik}a_{k} , i = 1 \dots m
                .\] 
                Если есть нулевой столбец альф, то а при этих альфах можем выкинуть и элементы b представляются k-1 элементом попали в индуктивное предположение. Нулевых столбцов нет. Мы можем предположить, что $\alpha_{11} \neq 0$
                \[
                    b_{2}' =  b_2 - \frac{\alpha_{21}}{\alpha_{11}}b_1
                .\] 
                \[
                b_3' = b_3  - \frac{\alpha_{31}}{\alpha_{11}}b_1
                .\] 
                Элементы b' зависимы по индуктивному предположению
                \[
                \gamma_2 b_2' + \gamma_3 b_3` + \dots \gamma_{m}b_{n}' = \overline{0}
                .\] 
         \end{enumerate}
     \end{proof}
\section{Базис и размерность линейного пространства. Переход к другому базису.}
Пусть у нас есть линейное пространство L над полем P.
\begin{definition}[Базис]
    Базисом линейного пространства L будем называть линейно независимую систему 
     $e_1,\dots,e_{n} \in L$ такую что $\forall  x\in L ~ x = \alpha_1 e_1 + \dots + \alpha_{n}e_{n}$ Альфы -- это координаты x по базису e.
     \[
     \alpha = (\alpha_1,\alpha_2,\dots,\alpha_{n})
     .\] 
     \[
     e = \begin{pmatrix} 
     e_1\\
     e_2\\
     \dots\\
     e_{n}
     \end{pmatrix} 
     .\] 
\end{definition}
\begin{theorem}
   Для выбраного элемента и базиса есть только один набор координат.
\end{theorem}
\begin{proof}
   Пусть существуют $x, e_1 \dots e_{n}$ такие что 
   \[
   x = \alpha_{1} e_{1} + \dots + \alpha_{n} e_{n}
   .\] 
\end{proof}
\begin{theorem}
   \[
   x = \alpha_{1} e_1 + \dots + \alpha_{n} e_{n}
   .\]  
   \[
   y = \beta_{1} e_1 + \dots + \beta_{n}y_{n}
   .\] 
   \begin{enumerate}
       \item $x + y = \sum_{i=1}^{n} (\alpha_{i} + \beta_{i})e_{i}$
      \item $\lambda \in P \lambda x = \sum_{i = 1}^{n} (\lambda x_{i})$
   \end{enumerate}
\end{theorem}
\begin{definition}[Размерность]
    Линейное пространство $L$  $(\dim{L} = n)$ Размерность L равна n если в этом пространстве существует система из n линейных независимых элементов, а любая система из n +1 линейно зависимая

    Если в линейном пространстве существует произвольное количество линейно независимых элементов то размерность бесконечная
\end{definition}
\begin{theorem}
    Любые n линейные независимые элементы n-мерного пространства образуют базис.
\end{theorem}
\begin{proof}
    С линейной назависимостью все ясно. Берем произвольный $x \in L$ добавили его в линейно независимую систему, получили линейно зависимую.
     \[
         \alpha_1 e_1 +  \dots \alpha_{n}e_{n} \alpha x= 0
    .\] 
    \[
    x = -\sum_{i = 1}^{n} \frac{\alpha_{i}}{\alpha}e_{i}
    .\] 
\end{proof}
\begin{theorem}
    Если есть базис из n элементов, то $\dim{L} = n$
\end{theorem}
\begin{proof}
    По последней теореме с прошлого занятия. Взяли $n + 1$ любых элементов из L, каждый из них раскладывается по базису.
\end{proof}
\begin{theorem}
    Любые два Линеныйных пространства одинаковой размерности над одним полем изоморфны.
\end{theorem}
\begin{proof}
    \[
        \dim{L} = \dim{L'} = n
    .\] 
    Назначим взаимно однозначное соответветсвие. У L базис $e_1,e_2,\dots,e_1$, У L' базис $e_1',e_2',\dots,e_{n}'$. Биекция между базисами есть.

    Берем $x \in L$ 
     \[
    x = \alpha_1 e_1 + \dots \alpha_{n} e_{n}
    .\] 
    \[
    x' = \alpha_{1} e_1' + \dots + \alpha_{n}e_{n}'
    .\] 
    Действительно является изоморфизмом.
\end{proof}
Пусть в линейном пространстве два базиса $e_1,\dots,e_{n} \land e_1'\dots e_{n}'$ 
\[
x = \alpha_{1}e_1 + \dots \alpha_{n} e_{n} = \alpha_1'e_1' + \dots + \alpha_{n}' e_{n}'
.\] 
\[
e_1' = p_{11}e_1 + p_{12}e_2 + \dots + p_{1n}e_{n}
.\] 
\[
e_{i}' = p_{i1}e_1 + p_{i2} + \dots + p_{in}e_{n}
.\] 
Из p можем собрать матрицу.
\[
e' = P e
.\] 
\[
    e = P'e'
.\] 
Обе матрицы обратимы, значит обратимы, значит ранги обоих матриц равны n.
 \[
x = \alpha e = \alpha' e'
.\] 
\[
\alpha e = \alpha' P e
.\] 
\[
\alpha = \alpha' P
.\] 
\[
\alpha' = \alpha P^{-1}
.\] 
\end{document}
