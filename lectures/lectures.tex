\documentclass{scrartcl}
\usepackage[utf8]{inputenc}
\usepackage[T2A]{fontenc}
\usepackage[russian]{babel}
\usepackage{amssymb}
\usepackage{amsmath}
\usepackage{amsthm}
\usepackage{listings}
\newtheorem{theorem}{Теорема}
\newtheorem{definition}{Определение}
\newtheorem{corollary}{Следствие}[theorem]
\newtheorem{lemma}[theorem]{Лемма}
\title{Лекции по алгебре}
\author{Титилин Александр}
\date{}
\begin{document}
\maketitle
\section{Линейное пространство, свойства, примеры}
Есть множество L = $\{a,b,\dots\}$ и есть некоторое поле  $P$
\begin{definition}
	L называется линейным пространством над полем P, если выполняются следущие условия
	\begin{enumerate}
		\item $\forall a,b \in L ~ \exists ! c : a + b = c$
		\item $\forall a,b,c \in L : (a + b) + c = a + (b + c)$
		\item $\exists  \overline{0} \in L \forall  a \in L a + 0 = a$
		\item $\forall  a\in L \exists  \overline{a} \in L : a +  \overline{a} = \overline{0}$
		\item $a + b = b + a \forall  a,b \in L$
	\end{enumerate}
	И существует операция $\forall  a\in L \forall  \alpha \in P \exists ! d \in L : \alpha a = d$
	\begin{enumerate}
		\item $\alpha , \beta \in  P a \in L = (\alpha + \beta)a = \alpha a + \beta a$
		\item  $\forall  a,b \in L , \forall \alpha \in P \alpha(a + b) = \alpha a + \alpha b $
		\item $\exists  1 \in P 1a = a \forall a \in L$
		\item $\alpha(\beta a) = (\alpha \beta)a \forall a \in L \forall \alpha ,\beta \in P $
	\end{enumerate}
\end{definition}
\subsection{Свойства операций}
\begin{enumerate}
	\item $\exists ! \overline{0}$
	      Пусть есть два ноля тогда их сумма может быть и первым и вторым значит они равны.
	\item $\forall  a\in L \exists ! \overline{a}$
	      Предполагаем что для какого-то а обратный не единственный, найдется два обратных $\overline{a_1} + a + \overline{a_2}$
	      \[
		      \overline{a_1} + (a + \overline{a_2}) = a_1 + \overline{0} = a_1
		      .\]
	      \[
		      (\overline{a_1} + a) + \overline{a_2} = (a + \overline{a_1}) + \overline{a_{2}} = \overline{0} + \overline{a_2} = \overline{a_2} + \overline{0} = \overline{a_2}
		      .\]
	\item $a , b \in L \exists  c \in L : a  = b + c$ c называется разностью $a,b$
	      Докажем что разность существует для любых двух элементов и она единсвенная.
	      $(a + \overline{b}) + b = a + (\overline{b} + b) = a + (b + \overline{b}) = a + \overline{0} = a$
	\item $0a = \overline{0} \forall \in L$
	      \[
		      a + 0a = 1a + 0a = (1 + 0)a = 1a = a
		      .\]
	\item $(-1)a = \overline{a} \forall  a\in L$
	      \[
		      a + (-1)a = 1a + (-1)a = (1 + (-1))a = 0a = 0
		      .\]
\end{enumerate}
\subsection{Примеры}
\begin{enumerate}
	\item Векторы на плоскости и в пространстве.
	\item Векторы с $n$ координат.
	\item Матрицы из вещ чисел  $n \times m$
	\item  $\{\overline{0}\}$ = L, $P = \mathbb{R}$
	\item $L =R_{+} ~ P = \mathbb{R} ~ a + b = ab ~ \alpha a = a^{\alpha}$
	\item L - множество всех полиномов степени не старше n $(P_{n}(t))$.
	\item L - множество всех функций непрерывных на отрезке от 0 до 1.
\end{enumerate}
Мы будем терминологически разделять компклесные и вещественные линейные пространства.
\begin{definition}
	Есть поле P, повесим над ним два линейных пространства $L, L'$. Эти два линейных пространства
	изоморфны друг другу, если между элементами существует биекция,такая что
	\begin{enumerate}
		\item $(a + b)' = a' + b' \forall  a, b \in L$ Образ суммы равен сумме образов.
		\item $(\alpha a)' = \alpha a', \alpha \in P$
	\end{enumerate}
\end{definition}
\section{Линейная зависимость, базис, размерность}
Рассмотрим $a_1,a_2,\dots,a_{n} \in L$
\begin{definition}[Линейная комбинация]
	\[
		\alpha_1 a_1 + \alpha_2a_2 + \dots + \alpha_{n}a_{n} = \sum_{i=1}^{n} \alpha_{i} a_{i}
		.\]
\end{definition}
Если линейная комбинация равна нулю, то она называется нулевой или тривиальной.
\begin{definition}[Линейная независимость]
	Система элементов являются линейно независимыми, если их нулевая линейная комбинация достигается при всех нулевых коэффициентов.
\end{definition}
\begin{definition}[Линейная зависимость]
	Если нулевую комбинацию можно получить, имея не нулевые коэффициенты, то она
	линейно зависимая
\end{definition}
\subsection{Примеры}
\begin{enumerate}
	\item Векторы на плоскость $\vec{v_1} = \{0,1\} \vec{v_2} =\{1,0\}$
	      \[
		      \alpha(1,0) + \beta(0,1) = (\alpha,\beta) = 0 \implies \alpha = 0 \land \beta = 0
		      .\]
\end{enumerate}
\subsection{Свойства}
\begin{enumerate}
	\item Если в системе элементов есть нейтральный элемент, то она линейно зависимая.
	\item Если в системе есть два одинаковых, то она линейно зависимая.
	\item Подномножество совокупность линейно зависимое $\implies$ вся совокупность линейно зависимая.
	\item Все подмножества линейно независимой системы линейно независимые.
\end{enumerate}
\begin{theorem}
	Система элементов линейно зависимой $\iff$ хотя бы один из элементов представлял собой линейную комбинацию остальных элементов.
\end{theorem}
\begin{proof}
	Необходимость. $\alpha_{i} \neq 0$ $\alpha_{i}a_{i} = -\alpha_1 a_{1} - \dots - \alpha_{n} a_{n}$
	Достаточность $a_{i} = \beta_1 a_{1} + \dots + \beta_{n} a_{n}$
\end{proof}
\begin{theorem}
	\[
		a_1,\dots,a_{k} \in L, b_1,\dots,b_{m} \in L
		.\]
	Любой элемент второй системы можно представить как линейную комбинацию первой.
	$m > k \implies $ элементы b линейно зависимы
\end{theorem}
\begin{proof}
	\begin{enumerate}
		\item $k = 1$
		      \[
			      b_1 = \alpha_{1} a_{1}
			      .\]
		      \[
			      b_2 = \alpha_{2}a_1
			      .\]
		      \[
			      \dots
			      .\]
		      \[
			      b_{m} = \alpha_{m} a_1
			      .\]
		      \[
			      -\alpha_2 b_1 + \alpha_1 b_2 + 0b_3 + \dots + 0 b_{m} = 0
			      .\]
		\item
		      Утверждение теоремы верно для $k - 1$
		      \[
			      b_1 = \alpha_{11} a_1 + \alpha_{12} a_{2} + \dots + \alpha_{1k}a_{k}
			      .\]
		      \[
			      b_{i} = a_{i1}a_1 + \dots a_{ik}a_{k} , i = 1 \dots m
			      .\]
		      Если есть нулевой столбец альф, то а при этих альфах можем выкинуть и элементы b представляются k-1 элементом попали в индуктивное предположение. Нулевых столбцов нет. Мы можем предположить, что $\alpha_{11} \neq 0$
		      \[
			      b_{2}' =  b_2 - \frac{\alpha_{21}}{\alpha_{11}}b_1
			      .\]
		      \[
			      b_3' = b_3  - \frac{\alpha_{31}}{\alpha_{11}}b_1
			      .\]
		      Элементы b' зависимы по индуктивному предположению
		      \[
			      \gamma_2 b_2' + \gamma_3 b_3` + \dots \gamma_{m}b_{n}' = \overline{0}
			      .\]
	\end{enumerate}
\end{proof}
\section{Базис и размерность линейного пространства. Переход к другому базису.}
Пусть у нас есть линейное пространство L над полем P.
\begin{definition}[Базис]
	Базисом линейного пространства L будем называть линейно независимую систему
	$e_1,\dots,e_{n} \in L$ такую что $\forall  x\in L ~ x = \alpha_1 e_1 + \dots + \alpha_{n}e_{n}$ Альфы -- это координаты x по базису e.
	\[
		\alpha = (\alpha_1,\alpha_2,\dots,\alpha_{n})
		.\]
	\[
		e = \begin{pmatrix}
			e_1   \\
			e_2   \\
			\dots \\
			e_{n}
		\end{pmatrix}
		.\]
\end{definition}
\begin{theorem}
	Для выбраного элемента и базиса есть только один набор координат.
\end{theorem}
\begin{proof}
	Пусть существуют $x, e_1 \dots e_{n}$ такие что
	\[
		x = \alpha_{1} e_{1} + \dots + \alpha_{n} e_{n}
		.\]
\end{proof}
\begin{theorem}
	\[
		x = \alpha_{1} e_1 + \dots + \alpha_{n} e_{n}
		.\]
	\[
		y = \beta_{1} e_1 + \dots + \beta_{n}y_{n}
		.\]
	\begin{enumerate}
		\item $x + y = \sum_{i=1}^{n} (\alpha_{i} + \beta_{i})e_{i}$
		\item $\lambda \in P \lambda x = \sum_{i = 1}^{n} (\lambda x_{i})$
	\end{enumerate}
\end{theorem}
\begin{definition}[Размерность]
	Линейное пространство $L$  $(\dim{L} = n)$ Размерность L равна n если в этом пространстве существует система из n линейных независимых элементов, а любая система из n +1 линейно зависимая

	Если в линейном пространстве существует произвольное количество линейно независимых элементов то размерность бесконечная
\end{definition}
\begin{theorem}
	Любые n линейные независимые элементы n-мерного пространства образуют базис.
\end{theorem}
\begin{proof}
	С линейной назависимостью все ясно. Берем произвольный $x \in L$ добавили его в линейно независимую систему, получили линейно зависимую.
	\[
		\alpha_1 e_1 +  \dots \alpha_{n}e_{n} \alpha x= 0
		.\]
	\[
		x = -\sum_{i = 1}^{n} \frac{\alpha_{i}}{\alpha}e_{i}
		.\]
\end{proof}
\begin{theorem}
	Если есть базис из n элементов, то $\dim{L} = n$
\end{theorem}
\begin{proof}
	По последней теореме с прошлого занятия. Взяли $n + 1$ любых элементов из L, каждый из них раскладывается по базису.
\end{proof}
\begin{theorem}
	Любые два Линеныйных пространства одинаковой размерности над одним полем изоморфны.
\end{theorem}
\begin{proof}
	\[
		\dim{L} = \dim{L'} = n
		.\]
	Назначим взаимно однозначное соответветсвие. У L базис $e_1,e_2,\dots,e_1$, У L' базис $e_1',e_2',\dots,e_{n}'$. Биекция между базисами есть.

	Берем $x \in L$
	\[
		x = \alpha_1 e_1 + \dots \alpha_{n} e_{n}
		.\]
	\[
		x' = \alpha_{1} e_1' + \dots + \alpha_{n}e_{n}'
		.\]
	Действительно является изоморфизмом.
\end{proof}
Пусть в линейном пространстве два базиса $e_1,\dots,e_{n} \land e_1'\dots e_{n}'$
\[
	x = \alpha_{1}e_1 + \dots \alpha_{n} e_{n} = \alpha_1'e_1' + \dots + \alpha_{n}' e_{n}'
	.\]
\[
	e_1' = p_{11}e_1 + p_{12}e_2 + \dots + p_{1n}e_{n}
	.\]
\[
	e_{i}' = p_{i1}e_1 + p_{i2} + \dots + p_{in}e_{n}
	.\]
Из p можем собрать матрицу.
\[
	e' = P e
	.\]
\[
	e = P'e'
	.\]
Обе матрицы обратимы, значит обратимы, значит ранги обоих матриц равны n.
\[
	x = \alpha e = \alpha' e'
	.\]
\[
	\alpha e = \alpha' P e
	.\]
\[
	\alpha = \alpha' P
	.\]
\[
	\alpha' = \alpha P^{-1}
	.\]
\subsection{Пример}
Есть плоскость, выбрали стандартный базис $e_1 = (0,1), e_2 = (1,0)$. Возьмем в качестве нового базиса эти векторы, повернутые на угол  $\varphi$
Нужна матрица преобразования.
\[
	e_1' = \cos{( \varphi )} e_1 + \cos{( \varphi )} e_2
	.\]
\[
	e_2' = -\sin{(\varphi)} e_1 + \cos{(\varphi)} e_2
	.\]
\[
	P =
	\begin{pmatrix}
		\cos{\phi}  & \sin{\phi} \\
		-\sin{\phi} & \cos{\phi}
	\end{pmatrix}
	.\]
\section{Подпространства и линейная оболочка}
\begin{definition}[Подпространство]
	L -- линейное пространство над $\mathbf{P}$.  $L_1 \subset L$ называется подространством пространства L если $\forall  x, y \in L_1 \forall  \alpha,\beta \in \mathrm{P}
		~ \alpha x + \beta y \in L_1$
\end{definition}
Любое линейное подпространство само является линейным пространством.

В пространстве выбрали подпространство $L_1$ ее базис образует линейно незавсимый набор элементов относительно $L$. Ее можно дополнить до базиса $L$
\subsection{Примеры}
\begin{enumerate}
	\item Векторы на плоскости, которые лежат на осях.
	\item Многочлены $P^{0}_{n}(t)$ -- многочлены, которые в нуле имеют значение 0.
\end{enumerate}
Задача на подумать. Размерность пространства полиномов, у которых сумма коэфициентов ноль.
\begin{definition}[Линейная оболочка.]
	В линейном пространстве L выбираем произвольное количество элементов. Линейной оболочкой этих элементов будем называть множество всех линейных комбинаций из этих элементов
	\[
		x,y,\dots, \in L
		.\]
	\[
		\alpha x + \beta y + \dots ~ \alpha,\beta \in \mathbf{P}
		.\]
	$x,y \dots$ система образующих
\end{definition}
\subsection{Пример}
\begin{enumerate}
	\item Комплексные числа -- это линейная оболочка над $1,i$
\end{enumerate}
\begin{theorem}
	Линейная оболочка является подпространством.
\end{theorem}
\begin{theorem}
	Линейная оболочка является наименьшим подпространством содержащим систему $x,y,\dots$

	Любое линейное подпространство, которое содержит $x,y,z,\dots$, будет содержать оболочку.
\end{theorem}
\begin{theorem}
	Размерность линейной оболочки равна количеству линейно независимых векторов в ее образующей системе.
\end{theorem}
\begin{proof}
	Размерность совпадает с количеством элементов в базисе. Взяли систему, нашли самое большой линейно-независмое подмножество. Остальные элементы это линейная комбинация этих элементов.
\end{proof}
\begin{theorem}
	Размерность линейной оболочки равна рангу матрицы, составленной из координат элементов образующей системы в произвольном базисе L.
\end{theorem}
\begin{proof}
	\[
		x,y,z \dots \in L
		.\]
	Каждый раскладываем по базису.
\end{proof}
\section{Линейная независимость относительно подпространства. Сумма и пересечение подпространств.}
\begin{definition}
	L -- линейное пространство, в этом линейном пространстве есть линейное подпространство $L_1 \subset L$. $v_1,\dots v_{k} \in L$
	Мы будем говорить, что эти элемены линейно независимые относительно $L_1$ , если
	\[
		\alpha_1 v_1 + \dots + \alpha_{k} v_{k} \in L_1
		.\]
	выполняется, только если все альфы нули.
\end{definition}
\begin{theorem}
	$L_1 \subset L, u_1\dots u_{m}$ базис. Для того чтобы вектора $v_1,\dots,v_{k}$
	Были линейно независимы относительно $L_1$ $\iff$  $u,v$ была линейно независимой
\end{theorem}
\begin{proof}
	Прямой ход. Создаем линейную комбинацию относительно
	\[
		\alpha_1 v_1 + \dots \alpha_{k} v_{k} + \beta_1 u_1 + \dots + \beta_{m} u_{m} = 0
		.\]
	\[
		\alpha_1 v_1 + \dots + \alpha_{k}v_{k} = -\beta_1 u_1 - \dots - b_{m}u_{m} \in L_1
		.\]
	Левая комбиация это ноль, так как линйно независима относительно $L_1$, правая линейно независима

	Обратный предположим, что $\alpha_1 v_1 + \dots + \alpha_{k}v_{k} \neq 0 \in L_1$.
	\[
		\alpha_1v_1 + \dots \alpha_{k} v_{k} = \gamma_{1}u_{1} + \dots \gamma_{m}u_{m}
		.\]
	\[
		\sum_{i=1}^{k}\alpha_{i}v_{i} - \sum_{i = 1}^{m}\gamma_{i} u_{i} = 0
		.\]
	Противоречие.
\end{proof}
\begin{definition}[Базис относительно подпространства]
	$v_1,\dots,v_{k} \in L$ будем называть базисом относительно $L_1 \subset L$
	если
	\begin{enumerate}
		\item $v_1,\dots ,v_{k}$ линейно независимы относительно $L_1$
		\item $\forall  x \in L  ~ x = \alpha_1v_1 + \dots \alpha_{k}v_{k} + y ~ y\in L_1$
	\end{enumerate}
\end{definition}
\begin{theorem}
	Для того, чтобы система была базисом была относительно $L_1$ необходимо и достачно
	система $u,v$ была базисом  $L$
\end{theorem}
\begin{proof}
	Необходимость. у представляем как линейную комбинацию $u$.
	Достаточность самим.
\end{proof}
\[
	L_{j}\subset L_{j - 1}\subset \dots L_2 \subset L_1 \subset L
	.\]
\begin{theorem}
	Любую систему линейно независимую относительно $L_1$ до базиса L относительно $L_1$
\end{theorem}
\begin{proof}
	Есть система линейно независимая относительная $L_1$. u,v -- линейно независимая совокупность дополним до базиса L. Получили базис относительно $L_1$, так как у нас выходит v, что дополнили + сумма из  $L_1$
\end{proof}
\section{Сумма и пересечение подпространств.}
Есть линенйное пространство L, $L_1,L_2$  его подпространства
\begin{definition}
	\[
		L_1 + L_2 := \{z  \in L | z = x + y, x \in L_1, y \in L_2\}
		.\]
\end{definition}
\begin{definition}
	\[
		L_1 \cap L_2 := \{x \in L | x \in L_1,x\in L_2\}
		.\]
\end{definition}
\begin{theorem}
	Сумма и пересечение подпространств сами являются подпросранствами
\end{theorem}
\begin{theorem}
	\[
		\dim{L_1} + \dim{L_2} = \dim{(L_1 +  L_2)} + \dim{(L_1 \cap L_2)}
		.\]
\end{theorem}
\begin{proof}
	\[
		\dim{L_1} = p
		.\]
	\[
		\dim{L_2}  = g
		.\]
	\[
		\dim{(L_1 + L_2)} = s
		.\]
	\[
		\dim{L_1 \cap L_2} = t
		.\]
	\[
		L_1 : u_1,\dots ,u_{p} v_1 \dots v_{p - t} \text{базис}
		.\]
	\[
		L_2 : u_1 \dots u_{t} w_1 \dots w_{g - t}
		.\]
	\[
		\alpha u + \beta v  + \gamma w =  0
		.\]
	\[
		\alpha u + \beta v  = -\gamma w = \nu u
		.\]
	Левая штука из $L_1$ , правая из $L_2$. Они из пересечения. Через $\nu$ разложили в базисе пересечения
	\[
		\nu u + \gamma w = 0
		.\]
	Получили нулевую линейную комбинацию базиса $L_2$
	\[
		\alpha u + \beta v = 0
		.\]
	\[
		z \in L_1 + L_2
		.\]
	\[
		t = x + y
		.\]
	x, y разложили по базису
\end{proof}
\section{Примеры}
\begin{enumerate}
	\item $L$ -- трехмерные вектора.
\end{enumerate}
\section{Алгоритм}
Берутся векторы, которые не вошли в базис суммы, раскладываеются по базису суммы.Теперь берем базисный вектор пересечения (часть разложения, где только векторы из $L_1$)
\section{Прямая сумма}
\[
	\exists  ! x_1 \in L_1 , x_2 \in L_2 : x = x_1 + x_2
	.\]
Множесвто таких иксов называется прямая сумма. $L_1 \oplus L_2$
\begin{theorem}[]
	\[
		L = L_1 \oplus L_2 \iff
		.\]
	\begin{enumerate}
		\item $L_1 \cap L_2 = \{ \emptyset \}$
		\item $\dim{L} = \dim{L_1} + \dim{L_2}$
	\end{enumerate}
\end{theorem}
\begin{proof}
	Предположим, что в пересечение есть не только нейтральный элемент.
	\[
		\exists x \neq 0 , x \in L_1 \cap L_2
		.\]
	\[
		x = x + 0 = 0 + x
		.\]
	Два разложения по $L_1,L_2$
	В другую сторону. Есть базис в $L_1$, есть базис в  $L_2$
	\[
		L_1 : u_1,\dots,u_{k}
		.\]
	\[
		L_2 : v_1,\dots,v_{m}
		.\]
	Базис $L$ можно соствить из объединения этих базисов.
	\[
		\sum \alpha u + \sum \beta v = 0
		.\]
	\[
		\sum \alpha u = - \sum \beta v
		.\]
	Элемент из $L_1$ равен элементу из $L_2$, тоесть они из $L_1 \cap L_2 = \emptyset$
	произвольный элемент разложили по базису из L. Получили определение прямой суммы
\end{proof}
\begin{theorem}
	Для того чтобы $L = L_1 \oplus L_2 \iff $ объединение базиса $L_1$ $L_2$ будет базисом L.
\end{theorem}
\begin{definition}
	\[
		L_1 \subset L
		.\]
	\[
		L^{\nu}_{1} - \text{Дополнение до L}
		.\]
	\[
		L_1 \oplus L_1^{\nu} = L
		.\]
\end{definition}
\section{Евклидовы и унитарные пространства}
У нас есть линейное пространство $E$ над полем $P$.
\begin{definition}[Евклидовое пространство]
	$E$ -- евклидово пространство.
	\[
		\forall  x,y \in E ~ \exists  (x,y) \in \mathbb{R}
		.\]
	Ставится в соответсвие вещественное число, которое назовем скалярным произведением. Это отображение удовлетворяет следущим пунктам
	\begin{enumerate}
		\item $(x,y) = (y,x) ~ \forall  x, y \in E$
		\item $(x + y,z) = (x,z) + (y,z) ~\forall  x,y,z$
		\item $(\lambda x,y) = \lambda(x,y) ~ \forall  x,y \in E, \forall  \lambda \in P$
		\item $(x,x) \ge 0 \forall  x \in E (x,x) = 0 \iff x = 0$
	\end{enumerate}
\end{definition}
\subsection{Свойства.}
\begin{enumerate}
	\item $(x,y+z) = (x,y) + (x,z)$
	      \[
		      (y + z,x ) = (y,x) + (z,x) = (x,y) + (x,z)
		      .\]
	\item $(x,\lambda y) = \lambda (x,y)$
	\item $(0,x) = 0 $
	\item $( \sum \alpha_{i} u_{i},\sum \beta_{j} v_{j} ) = \sum_{i}\sum_{j}\alpha_{i}b_{j}(u_{i},v_{j})$
	\item
	      \begin{theorem}[Неравенство Коши-Буняковского]
		      $\forall x,y \in E ~ (x,y)^{2} \le (x,x)(y,y)$
	      \end{theorem}
	      \begin{proof}
		      Если x или y равны 0, то се понятно. Рассмотрим
		      \[
			      \lambda x - y
			      .\]
		      \[
			      (\lambda x - y,\lambda x - y) = (\lambda x,\lambda x) - (y,\lambda x)
			      - (\lambda x, y) + (y,y) =
			      \lambda^2 (x,x) - 2\lambda (x,y) + (y,y) \ge  0
			      .\]
		      \[
			      D = (x,y)^2 - (x,x)(y,y) \le  0
			      .\]
	      \end{proof}
\end{enumerate}
\subsection{Унитарные пространства.}
\begin{definition}[Унитарные пространства]
	Линейное комплексное пространство называется унитарным, если любым двум элементам ставится в соответсвие комплексное число, которое называется скалярным произведением.
	$U$
\end{definition}
\subsubsection{Свойства}
\begin{enumerate}
	\item $(x,y) = \overline{(y,x)}$
	\item $(x+y,z) = (x,z) + (y,z) \forall  x,y,z \in U$
	\item $(\lambda x,y) = \lambda (x,y) , \forall  x,y \in U,\lambda \in P $
	\item $(x,x) \ge  0 ~\forall  x \in U, (x,x) = 0 \iff x = 0$
\end{enumerate}
\subsubsection{Примеры}
\[
	x = (\alpha_1,\dots,\alpha_{n})
	.\]
\[
	y = (\beta_1,\dots,\beta_{n})
	.\]
\[
	(x,y) = \sum \alpha_{i} \overline{\beta_{i}}
	.\]
\[
	(x,y + z) = \overline{(y + z,x)} = \overline{(y,x),(z,x)} = \overline{(y,z)} + \overline{z,x}=  (x,y) + (x,z)
	.\]
\[
	(x,\lambda x) = \overline{(\lambda y,x)} = \overline{\lambda} (x,y)
	.\]
\[
	(\sum \alpha_{i} u_{i},\sum \beta_{j} v_{j}) = \sum_{i}\sum_{j}
	.\]
\begin{theorem}
	\[
		|(x,y)|^2 \le  (x,x)(y,y)
		.\]
\end{theorem}
\begin{proof}
	\[
		(\lambda x - y , \lambda x - y) =
		|\lambda|^2(x,x) - \overline{\lambda} \overline{(x,y)} - \lambda(x,y) + (y,y) \ge  0
		.\]
	Рассмотрим частный случай
	\[
		\lambda = |\lambda| (\cos{\phi} - i \sin{\phi})
		.\]
\end{proof}
\begin{theorem}[Теорема Грама]
	\[
		u_1,\dots,u_{k} \in E
		.\]
	\[
		\Gamma =
		\begin{vmatrix}
			(u_1,u_1)   & (u_1,u_2)   & \dots & (u_1,u_{k})   \\
			\dots       & \dots       & \dots & \dots         \\
			(u_{k},u_1) & (u_{k},u_2) & \dots & (u_{k},u_{k})
		\end{vmatrix}
		.\]
	Вектора линейно зависимы
\end{theorem}
\begin{proof}
	Дано, что вектора зависимы
	\[
		\alpha_{1}u_1 + \dots + \alpha_{k} u_{k} = 0
		.\]
	\[
		\alpha (u_1,u_1) + \dots \alpha_{k} (u_1,u_{k}) = 0
		.\]
	Дальше так же умножили скалярно на остальные u. Получили однородную систему с ненулевым решением. Ее определитель ноль.\\
	В обратную сторону.
	Используем опеределитель как определитель системы
	\[
		\begin{cases}
			(u_1,u_1) \alpha_1 + \dots + (u_1,u_{k}) \alpha_{k} = 0 \\
			\dots                                                   \\
			(u_{k},u_1)\alpha_1 + \dots + (u_{k},u_{k}) \alpha = 0
		\end{cases}
		.\]
	\[
		\begin{cases}
			(\alpha_1 u_1,\alpha_1 u_1 + \dots + \alpha_{k} u_{k}) = 0 \\
			\dots                                                      \\
			(\alpha_{k} u_{k},\alpha_1 u_1 + \dots +  \alpha_{k} u_{k})
		\end{cases}
		.\]
	Набор альф -- ненулевое решение.
	Все сложили
	\[
		(\sum \alpha u, \sum \alpha u)
		.\]
	\[
		\sum \alpha u = 0
		.\]
\end{proof}
\begin{definition}[Нормированный]
	$x \in E$ нормированный, если  $(x,x) = 1$
\end{definition}
\begin{theorem}
	Любой вектор кроме нуля, можно нормировать.
\end{theorem}
\begin{proof}
	\[
		(x,x) = \lambda
		.\]
	\[
		\frac{(x,x)}{\lambda} = 1
		.\]
	\[
		(\frac{x}{\sqrt{\lambda} },\frac{x}{\sqrt{\lambda}}) = 1
		.\]
\end{proof}
\section{Ортогональность векторов и подпространств.}
По умолчанию живем в евклидовом пространстве.
\begin{definition}[Ортогональность]
	$x,y \in E, x \perp y := (x,y) = 0$
\end{definition}
\begin{definition}[Ортогональная система]
	Систему векторов называем ортогональной, если все векторы попарно ортогональны и нет нуля.
\end{definition}
\begin{definition}[Ортонормированная система.]
	Если в системе ортогональной системе, все элементы нормированны, то она ортонормированная
\end{definition}
\begin{theorem}
	Ортогональная(ортонормированная) система,линейно независима
\end{theorem}
\begin{proof}
	Так как в определителе Грама, все нули, кроме главной диагонали.
\end{proof}
\begin{corollary}
	В n мерном евклидовом пространстве, ортогональная система не может содержать более n элементов.
\end{corollary}
\begin{corollary}
	Если в ортогональной системе n -- элементов. Она базис (ортогональный/ортонормированный базис).
\end{corollary}
Будем называть ортонормированный базис естественным.
\begin{definition}[Символ Кронекера]
	$e_1,\dots,e_{n}$ -- ортонормированный базис
	\[
		(e_{i},e_{j}) = \delta_{ij} =
		\begin{cases}
			1, i = j \\
			0, i \neq j
		\end{cases}
		.\]
\end{definition}
\begin{theorem}
	Всякую линейно независимую систему линейного пространства можно перевести в ортогональную линейными преобразованиями элементов
	\[
		x_1, \dots, x_{k} \in E
		.\]
\end{theorem}
\begin{proof}
	Индукция по количество элементов в изначальной системе. Система из 1 элемента все ясно.\\
	Предположим, что мы умеем взяв $k - 1$ элемент превратить в  $k - 1$ ортогональный элемент. Возьмем систему из  $k$ линейно независимых элементов. Взяли  $k -1$ по предположению превратили в ортогональную. 
     \[
    y_{k} = x_{k} + \alpha_1 y_1 + \dots + \alpha_{k-1} y_{k - 1}
    .\] 
    \[
        (y_{k},y_{i}) = 0 , i = 1,\dots,k-1
    .\] 
    \[
        0 = (y_{k},y_1) = (x_{k},y_1) + \alpha_1(y_1,y_1)
    .\] 
    \[
    \alpha_{1} = - \frac{(x_{k},y_{i})}{(y_1,y_1)}
    .\] 
    \[
    \alpha_{i} = -\frac{(x_{k},y_{i})}{(y_{i},y_{i})}
    .\] 
    \[
    y_1 =x_1
    .\] 
    \[
    y_2 = x_2 - \frac{(x_2,y_1)}{(y_1,y_1)}y_1
    .\] 
    \[
    y_{k} = x_{k} - \sum_{i = 1}^{k - 1} \frac{(x_{k},y_{i})}{(y_{i},y_{i})} y_{i}
    .\] 
\end{proof}
\begin{corollary}
    В любом евклидовом пространстве существует ортогональный(ортонормированный) базис.
\end{corollary}
\begin{corollary}
    Любая ортогональная система элементов может быть дополнена до ортонормированного базиса.
\end{corollary}
\subsection{Cвойства ортонормированного базиса.}
\[
E: e_1,\dots,e_{n}, (e_{i},e_{j}) = \delta_{ij}
.\] 
Ортонормированный базис
\[
x,y \in E
.\] 
\[
x = \sum \alpha e
.\] 
\[
y = \sum \beta e
.\] 
\[
    (x,y) = \sum_{i = 1}^{n} \sum_{j = 1}^{n} \alpha_{i} \beta_{j}(e_{i},e_{j})=
    \alpha_{i}\beta_{i} + \dots \alpha_{n}\beta{n}
.\] 
В унитарном пространстве тоже самое,  беты сопряженные.
\begin{enumerate}
    \item 
        \[
        x \in E
        .\] 
        \[
        x = \sum \alpha e_1
        .\] 
        \[
            (x,e_1) = \alpha_1
        .\] 
        \[
            (x ,e_{i}) = \alpha_{i} , 1 = 1\dots n
        .\] 
    \item
        \[
        e_1',\dots , e_{n}'
        .\] 
        Новый ортонормированный базис.
        \[
        e' = P e
        .\] 
        \[
        e = P^{-1}e'
        .\] 
        \[
        e_1' = p_{11}e_1 + p_{12}e_2 + \dots + p_{1n}e_{n}
        .\] 
        \[
        e_i' = p_{i1}e_1 + p_{i2}e_2 + \dots + p_{in}e_{n}
        .\] 
\end{enumerate}
\begin{definition}
    Ортогональная матрица
    P -- квадратная.
    \[
    P P^{T} = E
    .\] 
\end{definition}
\[
E' \subset E
.\] 
\begin{definition} [элемент ортогонален подпространству]
    $x \perp L_1 \subset E := x \perp y ~ \forall y \in L_1$
\end{definition}
\begin{definition}
    \[
    L_1 \perp L_2 := \forall  x \in L_1 ~ \forall x \in L_2 x \perp y
    .\] 
\end{definition}
\begin{theorem}
    \[
        L_1 \cap L_2 = {\emptyset}
    .\] 
\end{theorem}
\begin{theorem}
    \[
    x \perp L_1
    .\] 
    Необходимо и достаточно, чтобы x был ортогонален всем элементам некоторого базиса $L_1$.
\end{theorem}
\begin{proof}
    Необходимость понятно x ортогонален всем. Достаточность $y = \sum \alpha u$ 
    \[
        (x,y) = 0
    .\] 
\end{proof}
\begin{corollary}
    Необходимо и достаточно, чтоб базисы $L_1, L_2$ были ортогональны.
\end{corollary}
\begin{definition}
    Ортогональное дополнение $L_1 \subset E$
    \[
        L_1^{\perp} := \{y \in E \mid y \perp L_1\}
    .\] 
\end{definition}
\begin{theorem}
    Ортогональное дополнение -- линенейно пространства.
\end{theorem}
\subsubsection{Построение ортогонального дополнения}
Дополнили базис $L_1$ до базиса L $u_1,\dots,u_{k},w_{k+1},\dots,w_{n}$. Строим ортогональный базис применили Грама-Шмидта. $u_1',\dots,u'_{k},w'_{k + 1},\dots,w_{n}'$
Дубльвэ из дополнения\\
\[
y \in L^{\perp}
.\] 
\[
y = \sum \alpha u' + \sum \beta w'
.\] 
\[
    (y,u_1') = 0 (\forall  \alpha = 0)
.\] 
ДубльВэ базис $L_1^{\perp}$
\begin{theorem}
    \[
    L_1 \oplus L_1^{\perp} = E
    .\] 
\end{theorem}
\begin{theorem}
    \[
   (  L_1 + L_2 )^{\perp} = L_1^{\perp} \cap L_2^{\perp}
    .\] 
\end{theorem}
\begin{theorem}
    \[
        (L_1 \cap L_2)^{\perp} = L_1^{T} + L_2^{T}
    .\] 
\end{theorem}
\begin{theorem}
    \[
        (L_1^{\perp})^{\perp} = L_1
    .\] 
\end{theorem}
\begin{theorem}
    Любые два евклидовых пространства одной размерности изоиорфны
\end{theorem}
\section{Проекция перпендикуляр точка. Длины, углы, расстояния.}
\[
L \subset E
.\] 
\[
L_1 \oplus L_1^{\perp} = E
.\] 
\[
\forall  x \in ~E \exists !~t \in L_1,z \in L^{\perp}~x = y + z
.\] 
y -- проекция на $L_1$, z -- перпендикуляр.
\subsection{Алгоритм}
\[
L_1:u_1,\dots,u_{k}
.\] 
\[
y = \sum \alpha u
.\] 
\[
    (x,u_1) =(y + z,u_1) = (y,u_1) = \alpha (u_1,u_1) + \dots + \alpha_{k} (u_{k},u_1)
.\] 
\[
    (x ,u_{i}) =  \alpha_1 (u_1,u_{i}) + \dots \alpha_{i}(u_{i},u_{k})
.\] 
Определитель этой системы не нуль, есть одно решение.
\begin{theorem}
    \[
    y = - \frac{1}{\Gamma} 
    \begin{pmatrix} 
        \gamma & u_1\\
        \dots & \dots \\
        (x,u_1) & \dots & (x,u_{k}) \\
    \end{pmatrix} 
    .\] 
\end{theorem}
\begin{definition}
    Линейное пространство $L$ называется нормированным, если
    \[
    \forall  x \in L \exists! ~  ||x|| \in P
    .\]
    Называется нормой если выполняются следущие свойства
    \begin{enumerate}
        \item $||x|| \ge  0~ ||x|| = 0 \iff x = 0 \forall  x \in L$
        \item $||\lambda x|| = |\lambda| ||x||$
        \item $||x + y|| \le ||x|| + || y||$
    \end{enumerate}
\end{definition}
\begin{enumerate}
    \item $||x||_{2} = \sqrt{(x,x)} $ -- евклидова норма(норма 2)
    \item $||x||_{1}$ $=$ $\sum |\alpha_{i}|$ 
    \item $||x||_{\infty} = \max{\alpha_{i}}$
\end{enumerate}
\begin{definition}[Расстояние между двумя элементами]
    \[
    x,y \in E
    .\] 
    $\rho(x,y) := || x - y ||$
\end{definition}
\begin{enumerate}
    \item $\rho(x,y) \ge  0 ~ \rho(x,y) = 0 \iff x = y$
    \item $\rho(x,y) = \rho(y,x)$
    \item $\rho(x,y)  \le \rho(x,z) + \rho(y,z) $
\end{enumerate}
\begin{definition}[Углы]
    \[
    x,y \in E
    .\] 
    \[
        (\widehat{x,y}) \in [0, \pi]
    .\] 
    \[
        \cos{\widehat{( x,y )}} = \frac{(x,y)}{||x|| * ||y||}
    .\] 
    \[
    |(x,y)| \le  \sqrt{(x,x)}  \sqrt{(y,y)} 
    .\] 
\end{definition}
\begin{theorem}[Теорема косинусов]
    \[
        ||x - y|| ^{2} = (x-y,x-y) = ||x||^{2} + ||y||^{2} - 2||x|| *||y|| \cos{\widehat{( x,y )}}
    .\] 
\end{theorem}
\[
  (||x|| - ||y||)^{2} \le  ||x - y|| ^{2} \le  (||x|| + ||y||)^{2}
.\] 
\[
||x || - || y|| \le  ||x - y|| \le  ||x|| + ||y||
.\] 
\begin{definition}[Расстояние до подпространства]
    \[
        \rho(x,L_1) := \inf_{u \in L_1}{\rho(x,u)}
    .\] 
\end{definition}
\begin{definition}
    \[
        \rho( L_1,L_2 ) = \inf_{u \neq 0 \in L_1 , v \neq 0 \in L_2}~ \rho(u,v)
    .\] 
\end{definition}
\[
x \in E, L_1 \subset E
.\] 
\[
u \in L
.\] 
\[
    \rho(x,u) = \rho(y + z,u)= ||x - u||^{2} = ||y - u + z||^{2} = ||y - u||^{2} + ||z||^{2}
.\]
Минимум этой фигни, когда $y = u$, вот так ищются расстояния между подпространствами.
 \[
\rho(x,L_1) = ||z||
.\] 
\begin{definition}
    \[
        (\widehat{x.L_1}) = \inf_{u \in L_1} (\widehat{x,u})
    .\] 
\end{definition}
\[
    \cos{\widehat{x,y}} = \frac{(x,u)}{||x|| ||u||} = \frac{||y||}{||x||} \cos{\widehat{y,u}}
.\] 
Эта фигня достигает максимума при $u = y$
\[
    (\widehat{x,L_1}) = (\widehat{x,y})
.\] 
\[
a \perp L_1
.\] 
\[
    (x.a) = (z,a)
.\] 
\[
\frac{|(x,a)|}{||a||}
.\] 
\section{Операторы}
X , Y -- линейные пространства над полем P.
\begin{definition}[Операторы]
    \[
        \mathcal{A} : X \to Y
    .\] 
    Опреатором из X в Y мы будем называть отображение при котором
    \[
        y = \mathcal{A} x
    .\] 
    Оператор линейный если
    \[
    \forall  x_1, x_2 \in X , \forall  \alpha_{1} ,\alpha_{2} \in P
    .\] 
    \[
        \mathcal{A} (\alpha_{1} x_1 + \alpha_{2}x_2) = \alpha_1 \mathcal{A}x_1 +
        \alpha_2 \mathcal{A} x_{2}
    .\] 
\end{definition}
\subsection{Термины}
\begin{enumerate}
    \item Если оператор переводит в скаляр, то оператор  называется функционалом.
    \item  Если $Y = X$ говорим что оператор действует в X.(оператор преобразования X)
    \item  Нулевой оператор, если $Y = \{0\}$ ,  $\mathcal{O} x = 0$
    \item  $\mathcal{E} x = x  ~\forall  x \in X$
\end{enumerate}
\subsection{Примеры}
\begin{enumerate}
    \item $P_{n}(t)$
        \[
            \mathcal{A}: P_{n}(t) \to P_{n - 1}(t)
        .\] 
         \[
             \mathcal{A} f(t) = f'(t)
         .\] 
         Оператор диференцирования, обычно обозначается $\mathcal{D}$
         \item  $\mathcal{I} f(t) = \int f(t) dt + c$
             \[
                 \mathcal{I} : P_{n}(t) \to P_{n + 1}(t)
             .\] 
        \item Норма вектора -- функционал.
\end{enumerate}
\subsection{Операции с операторами}
Рассмотрим множество линейных опeраторов из X в Y.
\begin{enumerate}
    \item Операторы равны, если $\forall x \in \mathcal{X}~ \mathcal{A}x = \mathcal{B}x$
    \item Сложение операторов $\forall x \in \mathcal{X} ~( \mathcal{A} + \mathcal{B} ) x := \mathcal{A} x + \mathcal{B} x$
    \item $\forall  x \in \mathcal{A} \lambda \in P (\lambda \mathcal{A}) x := \lambda \mathcal{A} x$
\end{enumerate}
\begin{theorem}
    Сумма и умножение линейных операторов есть линейный оператор
\end{theorem}
\begin{proof}
    \[
        (\mathcal{A} + \mathcal{B}) (\alpha_1 x_1 + \alpha_{2} x_2) = 
    .\] 
\end{proof}
\begin{theorem}
    Множество линейных операторов и такие операции образуют линейное пространство. $L(X,Y)$
\end{theorem}
\subsection{Умножение операторов}
\[
    \mathcal{B} : X \to Y
.\] 
\[
    \mathcal{A} : Y \to Z
.\] 
$( \mathcal{A} \mathcal{B} )x = \mathcal{A} (\mathcal{B}x)$
\subsubsection{Свойства}
\begin{enumerate}
    \item линейность
    \item $(\mathcal{A} \mathcal{B}) \mathcal{C} = \mathcal{A} (\mathcal{B} \mathcal{C})$
    \item $(\mathcal{A} + \mathcal{B}) \mathcal{C} = \mathcal{A} \mathcal{C} + \mathcal{B} \mathcal{C}$
    \item  $\mathcal{A} (\mathcal{B} + \mathcal{C}) = \mathcal{A} \mathcal{B} + \mathcal{a} + \mathcal{C}$
    \item  $\lambda(\mathcal{A} \mathcal{B}) = (\lambda \mathcal{A}) \mathcal{B}$
    \item  $E \mathcal{A} =\mathcal{A} E = \mathcal{A}$
\end{enumerate}
Доказательсва дома
$L(X,X)$ -- образуют кольцо c 1.
\begin{theorem}
    Cтепени оператора коммутативны
\end{theorem}
\section{Ядро и образ, ранг и дефект}
Рассмотрим $\mathcal{A} \in L(X,Y)$
\begin{definition}[Ядро оператора]
    \[
        \ker{\mathcal{A}} := \{x \in X | \mathcal{A} x = 0\}
    .\] 
    \[
        \ker{\mathcal{A}} \subset X
    .\] 
\end{definition}
\begin{theorem}
    Ядро образует подпрострастно $X$.
\end{theorem}
\begin{definition}[Дефект]
    \[
        def \mathcal{A} := \dim{\ker{\mathcal{A}}}
    .\] 
\end{definition}
\begin{definition}[Образ]
    \[
        im{\mathcal{A}} := \{\mathcal{A} x | x \in X\}
    .\] 
\end{definition}
\[
    y_1,y_2 \in \im{\mathcal{A}}
.\] 
\[
\alpha_1 y_1 + \alpha_2 y_2 = \mathcal{ A }\alpha_1 x_1 + \alpha_2 x_2
.\] 
\begin{definition}[Ранг]
    \[
        r{\mathcal{A}} := \dim{im~ \mathcal{A}}
    .\] 
\end{definition}
\subsection{Cвойства}
\begin{enumerate}
    \item Дефект меньше или равен размерности $X$
        \[
            def 0 = \dim{X}
        .\] 
        \[
        def E = 0
        .\] 
    \item Ранг меньше или равен Y
\end{enumerate}
\begin{theorem}
    \[
        \forall  \mathcal{A} \in L(X,Y)
    .\] 
    \[
        rank \mathcal{A} + def \mathcal{A} = \dim X
    .\] 
\end{theorem}
\begin{proof}
    \[
        \ker{\mathcal{A}} : u_1,\dots,u_{k} - \text{Базис}
    .\] 
    \[
    u_1,\dots,u_{k} , v_1 \dots, v_{n - k}
    .\] 
    Дополнили базис ядра до базиса $X$.
    \[
        \mathcal{A} v_1 \dots \mathcal{A} v_{n - k}
    .\] 
    Построили из этой фигни линейную комбинацию нулевую и линейность оператора
    \[
        \sum \alpha \mathcal{A} v = 0
    .\] 
    Комбинация из ядра. Разложили по базису ядра приравняли, перенесли, получили линейная комбинация из базиса равна нулю.
    Таким образом
    \[
        \mathcal{A} v_1 \dots \mathcal{A} v_{n - k} ~ \text{Базис Образа}
    .\] 
    \[
        \exists  y \in Im \implies \exists  x \in X y = \mathcal{A}x
    .\] 
    x разложили по базису получили линейную комбинацию $\mathcal{A} v_1,\dots, \mathcal{A} v_{n - k} $
\end{proof}
\begin{theorem}
    \[
    A,B \in L(X,X)
    .\] 
    \[
    rank AB\le  rank A
    .\] 
    \[
    def A \le def AB
    .\] 
    \[
    rang B \le  rank B
    .\] 
\end{theorem}
\begin{theorem}
    \[
        rank AB \ge rang A + rang B - \dim{X}
    .\] 
\end{theorem}
\begin{proof}
    \[
    ker AB : u_1,\dots,u_{p},u_{p + 1}, \dots, u_{p+q}
    .\] 
    \[
    q = def b
    .\] 
    \[
        \sum \beta B u = 0
    .\] 
    \[
    B(\sum \alpha u ) = 0
    .\] 
\end{proof}
\section{Невырожденные операторы}
\begin{definition}
    \[
        \mathcal{A} \in L(X,X)
    .\] 
    Оператор называется невырожденым если в его ядре только нейтральный элемент.
    \[
        \ker{\mathcal{A}} = \{0\}
    .\] 
\end{definition}
\subsection{Свойства}
\begin{enumerate}
    \item $im \mathcal{A} = X$
    \item $\forall  x \in X \exists ! y = \mathcal{A} x$
    \item Произведение невырожденных операторов являетс невырожденным
\end{enumerate}
Давайте предоложим 
\[
    \exists  y \in X ~ y = \mathcal{A} x_1 , y = \mathcal{A} x_2
.\] 
\[
    0 = \mathcal{A} x_1 - \mathcal{A} x_2 = \mathcal{A}(x_1 - x_2) = 0
.\] 
\[
    x_1 - x_2 \in \ker \mathcal{A}
.\] 
\begin{definition}
    Рассмотрим  оператор из X в X.
    \[
        x = \mathcal{A}^{-1} y
    .\] 
    Обратный оператор
\end{definition}
\begin{theorem}
    Обратный оператор линейный 
\end{theorem}
\begin{proof}
    \[
        y_1 = \mathcal{A} x_1
    .\] 
    \[
        y_2 = \mathcal{A} x_2
    .\] 
    \[
        \alpha y_1 + \beta y_2 = \mathcal{A } \alpha y_1 + \mathcal{A } \beta y_2
    .\] 
    \[
    .\] 
\end{proof}
\begin{theorem}
    Обратный оператор невыражденный
\end{theorem}
\begin{proof}
    \[
        \mathcal{A}^{-1} y = 0
    .\] 
    \[
        \mathcal{A} \mathcal{A}^{-1} y  = \mathcal{A} 0 = 0
    .\] 
    \[
    y = 0
    .\] 
\end{proof}
\begin{theorem}
    \[
        \mathcal{A} \in L(X,X)
    .\] 
    \[
        \mathcal{A}^{2} = 0  \iff im \mathcal{A} \subset \ker A
    .\] 
\end{theorem}
\begin{theorem}
    \[
        \mathcal{P} x=  y
    .\] 
    \[
    x = y + z
    .\] 
    \[
    y \in L, z \in L^{\perp}
    .\] 
\end{theorem}
\section{Матрица операторов}
\[
    \mathcal{A} \in L(X,Y)
.\] 
\[
    X : e_1,\dots,e_n - \text{Базис}
.\] 
\[
    Y: q_1,\dots,q_{m} - \text{Базис}
.\] 
\[
x \in X
.\] 
\[
x = \sum \alpha_{j} e_{j}
.\] 
\[
    y = \mathcal{A} x = \sum \alpha_{j} \mathcal{A} e_{j}
.\] 
\[
 y = \sum  \beta_{i} q_{i}
.\]
\[
    \mathcal{A} e_1 = a_{11} q_1 + a_{21} q_2 + \dots + a_{m1} q_{m}
.\] 
\[
    \mathcal{A} e_2 = a_{12} q_1 + \dots a_{m 2} q_{m}
.\] 
\[
    \mathcal{A} e_{n} = a_{1 n} q_1 + a_{2 n} + \dots a_{mn} q_{m}
.\] 
Из а собрали матрицу
\[
    \mathcal{A} e_{j} = \sum_{i = 1}^{m} a_{ij} q_{i}
.\] 
\[
y = \sum_{i = 1}^{m} \beta_{i}q_{i} = \sum_{j = 1}^{n} \alpha_{j} ( \sum_{i = 1}^{m} a_{ij} q_{i} ) =
\sum_{i = 1}^{m} ( \sum_{j = 1}^{n} a_{ij} \alpha_{j} ) q_{i}
.\] 
\[
    \beta_{i} = \sum_{j = 1} ^{n} a_{ij} \alpha_{j}
.\] 
\[
\begin{pmatrix} 
\beta_1\\
\dots\\
\beta_{n}\\
\end{pmatrix}  =
A
\begin{pmatrix} 
\alpha_1\\
\dots\\
\alpha_{n}
\end{pmatrix} 
.\] 
\[
Y_{q} = A_{qe}X_{e}
.\] 
$A_{qe}$ -- матрица оператора в паре базисов e,q.
Размер А -- $\dim{X} \times \dim{Y}$
\end{document}
